\title{Notes: gradient of log-likelihood for Markov Chain choice}
\author{Sven Schmit \& Victor Minden}
\date{\today}

\begin{document}

    \maketitle
    \section{The model}
    In the pairwise-choice Markov chain model \cite{pcmc}, the log-likelihood is given by
    \begin{align*}
        \log \mathcal{L}(P; C) = \sum_{S \subset U} \sum_{i \in S} C_{iS} \log(p_{iS}(P))
    \end{align*}
	and we are interested in finding its derivative.
	We focus on a single term $\log(p_{iS}(P))$.
	


	A brief note on notation, we use the following shorthand to denote the derivative
	\[
		\dot x = \frac{\partial x}{\partial t}
	\]

    Lemma from \cite[Theorem 3.2]{golub}:
    \begin{lemma}
        If $P = P(t)$ is a row stochastic, irreducible matrix, and differentiable in $t$, then
        the derivative of the stationary distribution associated with $P$ is given by
        \[
			\dot \pi = \pi \dot P (I - P)^\dagger
        \]
        where $(I - P)^\dagger$ is the pseudo-inverse of $I - P$.
    \end{lemma}

    Let $Q$ be the rate matrix corresponding to a continuous time Markov chain.
    We define $q$ as the equilibrium distribution of $Q$, 
	and $\nu_i = \sum_{j \ne i} Q_{ij}$ the transition rate from state $i$ to other states.
    Let $P$ denote the embedded discrete jump chain of $Q$ and let $\pi$ denote its equilibrium distribution,
    that is $P$ is the transition matrix of a discrete time Markov chain with jump probabilities
    \[
        P_{ij} = \frac{Q_{ij}}{\sum_{k\ne i} Q_{ik}} = \frac{Q_{ij}}{\nu_i} \text{ for } i \ne j
    \]
	and $P_{ii} = 0$.

    Then we know that (\cite[page 398]{ross}) the equilibrium distribution of the continuous time
	Markov chain is related to equilibrium distribution of the embedded chain as
    \[
        q_i = \frac{\pi_i / \nu_i}{\sum_j \pi_j / \nu_j}
    \]

	Suppose $Q = Q(t)$, then first note that
	\[
		\dot P_{ij} = \frac{\nu_i \dot Q_{ij} - Q_{ij} \dot \nu_i}{\nu_i^2}
	\]
	and
	\[
		\dot \nu_i = \sum_{j \ne i} \dot Q_{ij}
	\]
	Define, for convenience,
	\[
		\alpha_i = \frac{\pi_i}{\nu_i}
	\]
	then its derivative with respect to $t$ is
	\[
		\dot \alpha_i = \frac{\nu_i \dot \pi_i - \pi_i \dot \nu_i}{\nu_i^2}
	\]
	Finally, we have
	\[
		\dot q_i = 
		\frac { \dot \alpha_i \sum_{j \ne i} \alpha_j - \alpha_i \sum_{j \ne i} \dot \alpha_j }
				{ \left( \sum_{j \ne i} \alpha_j \right)^2 }
	\]

    \bibliographystyle{abbrv}
    \bibliography{bib/bibliography}

\end{document}
